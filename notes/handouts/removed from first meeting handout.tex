\documentclass[12pt,letterpaper,table,svgnames,dvipsnames]{article}

% \begin{preamble}
\usepackage[margin=1in]{geometry}
% \usepackage{times}
\usepackage{helvet}
\renewcommand{\familydefault}{\sfdefault}

\usepackage{enumitem}
\usepackage{apacite}
% \usepackage{dblfloatfix}
\usepackage{caption}
\usepackage{subcaption}
\usepackage{graphicx}
\usepackage{tcolorbox}
% \usepackage{multicol}
\usepackage{makecell}
\usepackage{xcolor}
\usepackage{booktabs}
\usepackage{url}

\usepackage{graphicx}
\graphicspath{ {/images/} }

\usepackage{enumitem}%
% Nested enumeration with numbers rather than letters
\renewcommand{\labelenumii}{\theenumii}
\renewcommand{\theenumii}{\theenumi.\arabic{enumii}.}




\begin{document}





What is the Starting Point?
\begin{enumerate}
    \item Affective Primacy Hypothesis
    \item Meaning is $\{not / more than \}$ truth conditions
    \item The affective (qua, non-TC meaning) is lexically represented?
\end{enumerate}



\begin{itemize}
    \item 
\end{itemize}







And this is the thing that is going to happen...
it boils down to the representation of narrow content....two dimensional semantic accounts, but that doesn't solve the problem of the formalizm



\newpage





Is valence represented 


\begin{enumerate}[noitemsep]


    \item Zajonc (1980) and the affective primacy hyptohesis

    How important is affect/emotion to decision making?
    How important is emotion in object recognition?

    How important is emotion in word recognition?
    --->

    Zajonc's intuition was that Affect, is necessary for adaptive behavior



    \item Questions about Affective meaning:
         \begin{enumerate}[noitemsep]

            \item Is affect part of the literal meaning?
                \begin{enumerate}[noitemsep]%
                \item In light of affect / expressive meaning, can we maintain TC accounts of meaning?

                \item many years of TC menaing proponents who stumble in the face of this kind of data 
                        \begin{enumerate}[noitemsep]%
                            \item Davidson's ``derangement'' (1986)%
                            \item Kaplan, Kratzer on ``Oops'' and ``ouch''%
                            % \item 
                        \end{enumerate}
            \end{enumerate}
            
            \item Which notion of `affect' is relevant to the present study?
                \begin{enumerate}[noitemsep]
                    \item valence as an feature of lexical meaning?
                        $\rightarrow$ first, IS valence lexically represented?
                    \item valence as an indication of an affective (=physiological) response 
                    
                    \item valence = affective semantic knowledge?
                \end{enumerate}
            
            \item How is affective meaning (=affective semantic knowledge) psychologically represented?   
                \begin{enumerate}[noitemsep]
                    \item Is it part of the lexicon?
                        ...How is the lexicon represented? as part of long-term memory? How is long-term memory behaviorally distinguished from other kinds of memory?
                    \item 
                \end{enumerate}
            
        \end{enumerate}   
            
        \item Questions about non-Affective Meaning:
            \begin{enumerate}[noitemsep]
                \item What is the best way to categorize the non-affectual component of verb meaning?
                    \begin{itemize}[noitemsep]
                        \item Expressive/Non-TC meaning (a.o.t. affective dimension)
                        \item 
                    \end{itemize}
                \item Psychological/Physical vs. Abstract/Concrete
                    \begin{itemize}[noitemsep]
                        \item Löhr (2021), Dunn (2015)
                        \item effects of ambiguity, polysemy on concreteness/abstractions
                        \item France is a hexagon and a democracy / \* France is a hexagonal democracy
                    \end{itemize}
                \item Theories of abstract vs concrete concepts:
                    \begin{itemize}
                        \item \textbf{Dual-Coding theory} (Paivio 1991; Barsalou et al 2008)\\
                            \begin{itemize}
                                \item Abstract concepts are associated with fewer sensorimotor-introspective reprsentations, thus rely on other representational forms (language, dis-embodied); concrete concepts are represented both by language and sensorimortor-introspective information
                                \item Thus, concrete faster than abstract
                            \end{itemize}

                        \item \textbf{Context Availiability} (Schwanenflugel 1992)
                            \begin{itemize}
                                \item For abstract concepts, you need enough contextual/general world knowledge information to sufficiently characterize meaning (especially out of context)
                            \end{itemize}
                        \item \textbf{Metaphorical} (Lakoff \& Johnson 1999)
                            \begin{itemize}
                                \item Abstract concepts are represented via metaphorical mappings to concrete concepts
                                \item Thus, they rely on concrete concepts conceptually
                            \end{itemize}
                    \end{itemize}
            \end{enumerate}    
            
            \item What is meaning?
                \begin{itemize}[noitemsep]
                    \item Expressive/Non-TC meaning (a.o.t. affective dimension)
                    \item 
                \end{itemize}
                
            \item What is the goal of the study?
                \begin{enumerate}[noitemsep]
                    \item Is affect (=VALENCE) lexically encoded?
                    \item 
                \end{enumerate}
           
            
            
            \item How does affect interact with conceptual representations?
            \item Do conceptual representations *Always* trigger a physiological response?
            \item Does the valence of a lexical item incur (correspond) with a physiological response?


            
        \end{enumerate}



first the verbs need to be normed for the strength of their valence





\begin{enumerate}[noitemsep]
    \item What are the goals of this project?
        \begin{enumerate}[noitemsep]
            \item 
            \item 
            \item 
        \end{enumerate}
\end{enumerate}




\section{Groundwork}
First things first. Define what we mean when we say ``affect'': 



\section{Current State of Knolwedge}


\subsection{The relationship between Concreteness and valence: Abstract concepts are rated as more emotional than concrete ones}


The ``v-relationship''


Warriner et al (2013)



Traditional 


\begin{itemize}[noitemsep]
    \item abstract concepts are a heterogenous group
        % \begin{itemize}[noitemsep]
        %     \item 
        %     \item 
        %     \item 
        % \end{itemize}
\end{itemize}




\subsection{What we know from neuroscience}




\subsubsection{Early Posterior Negativity (EPN)}


\begin{itemize}
    \item Between 200-300 ms after stimulus onset, a negative deflection in visual cortices for emotion versus neutral words

    \item Associated with increated attention to emotional stimuli

    \item Observed for adjectives (Herbert, Kissler, Junghofer, Peyk, \& Rockstroh, 2006), nouns (Kissler et al., 2007), and verbs (Schacht \& Sommer, 2009a; Schacht \& Sommer, 2009b)

\end{itemize}

\subsubsection{Late positive complex (LPC)}


\begin{itemize}
    \item Increased centro-parietal possitivity

    \item associated with more elaborated semantic processing

\end{itemize}


\subsubsection{Caveates}

\begin{itemize}
    \item Is it a response to the stored semantic representation of emotional content (semantic valence) or is is a marker of the conditioned association between a stimulus and a some (emotional) visual feature of the word?

    \item Differences between syntactic category? Palazova at al (2013) versus Kanzke and Kotz (2007): the latter found no late interaction between emotion and concreteness for nouns, tho the former did for verbs...

\end{itemize}




\subsection{Summaries of papers suggested by reviewers}

\subsubsection{De Deyne et al (2014): Graded structure in adjective categories}

\begin{itemize}
    \item RQ: Are adjective categories graded? What determines their structure?
\end{itemize}


\subsubsection{Kousta et al (2011)}

\begin{itemize}
    \item 
\end{itemize}

\subsection{Palazova et al (2013): erp and verbs}


\begin{itemize}
    \item Valence versus concreteness in verbs?
    \item Behavioral responses in the lexical decision task were faster to concrete than to abstract words, and RTs \emph{increased} for positive/negative valenced abstract words compared to neutral concrete words.
\end{itemize}


\subsection{Winter (2022)}

\begin{itemize}
    \item 
\end{itemize}


\section{Suggested Experiments}

\subsection{Norming}

\begin{itemize}
    \item concreteness rating (replaces physical/psychological dimension)
    \item valance rating
    \item 
    \item semantic similarity 
\end{itemize}


\subsection{Evaluative Priming}


\begin{itemize}[noitemsep]
    \item Task Manipulation: Concreteness, Valence
        \begin{itemize}[noitemsep]
            \item Prime (+valence, -valence, 0 valence) x Condition (congruent,incongruent)
            \item Prime (concrete, abstract) x Condition (congruent,incongruent)
        \end{itemize}
    \item Dependent variable: Response time, error rates
    \item Predictions:
        \begin{itemize}[noitemsep]
            \item H1: 
        \end{itemize}
\end{itemize}


\subsection{Statistical Model}


\begin{itemize}
    \item 
\end{itemize}



\newpage


\noindent Summary of Reviews
\begin{enumerate}
    \item \mm{Review: More thorough cover of psychological background literature on affect and meaning needed}\\
    \mm{I agree with this, surveying the psychological and neuroscience literature paints a complex picture of emotion and its interaction with language}
    
 


    \mm{Raises a Conceptual question...what are the real goals of this project? the ``importance of affective meaning''}

        \begin{enumerate}
            \item Vindicate alternative (qua: non-truth conditional) accounts of meaning?

            \item Test whether valence is lexically represented?
            % \item Test 
        \end{enumerate}

    \mm{A linking theory is needed to bridge the studies and the theoretical literature.}

    \item \mm{Review: Suggested items in the lit review}:

        \begin{itemize}
            \item Kousta et al (2009)
            \item Kousta et al (2011)\\
            Are abstract concepts grounded in emotion (read: affective features)?
            % \item Vigliocco et al XXXX
            \item De Deyne 2014
            \item Ponz et al 2014
            \item Winter (2023)
        \end{itemize}

        Actually, I'm not sure these are even as relevant as they seem, except for Winter (2023). The others are already somewhat outdated since the science has progressed signficantly but also there are many conflicting results and the final picture is a little complex.

        \textbf{Hinojosa, Moreno \& Ferré (2019), ``Affective Neurolingusitics''}


    




    \item \mm{Review: What counts as evidence for Affect First Hypothesis? whether ``psychological domain'' counts as the relevant non-affect comparison}
        
        \begin{enumerate}[noitemsep]
            \item It could be easier (=faster) to judge valence wrt verbs more than concreteness \\

            \mm{One response: Compare the reaction times of the two tasks, do participants take longer to judge concreteness than valence in general during the norming?}
        \end{enumerate}


        \begin{enumerate}
            \item The APH is actually underdetermined wrt to 
            \item there are several different affect responses....
        \end{enumerate}

   \item \mm{Review: Is``psychological domain'' counts as the relevant non-affect comparison}

    \item \mm{Review: Debate about what Frege actually meant}



    \item Misc Points:
        
        \begin{enumerate}
            \item Are abstract concepts grounded in emotion meaning features ? do we find differences in domain categories? (cf. Kousta et al 2011)

            \item embodiment more generally?
        \end{enumerate}



\end{enumerate}


\end{document}