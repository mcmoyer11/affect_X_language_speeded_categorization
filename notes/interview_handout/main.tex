\documentclass[12pt,letterpaper,table,svgnames,dvipsnames]{article}

% \begin{preamble}
\usepackage[margin=1in]{geometry}
% \usepackage{times}
\usepackage{helvet}
\renewcommand{\familydefault}{\sfdefault}

\usepackage{enumitem}
\usepackage{apacite}
% \usepackage{dblfloatfix}
\usepackage{caption}
\usepackage{subcaption}
\usepackage{graphicx}
\usepackage{tcolorbox}
% \usepackage{multicol}
\usepackage{makecell}
\usepackage{xcolor}
\usepackage{booktabs}
\usepackage{url}
\usepackage{enumitem}%

% Nested enumeration with numbers rather than letters
\renewcommand{\labelenumii}{\theenumii}
\renewcommand{\theenumii}{\theenumi.\arabic{enumii}.}

% % To keep track of total pages
% \usepackage{fancyhdr}
% \usepackage[page]{totalcount}
% \pagestyle{fancy}
% \fancyhf{}
% \cfoot{Page \thepage~of \totalpages}

\usepackage{gb4e}
\noautomath
% \end{preamble}

\definecolor{Purple}{RGB}{255,10,140}
\newcommand{\ad}[1]{\textcolor{Purple}{[ad: #1]}}
\newcommand{\mm}[1]{\textcolor{teal}{[mm: #1]}}

\begin{document}
% \begin{center}
\noindent \Large\textbf{Comments on Bourmayan, Stojanovic \& Strickland (2023) ``Valence First''}
\bigskip

\large 
\noindent Morgan Moyer \\ 
\normalsize
\noindent \today \\
% \end{center}

\hrule

\bigskip
\bigskip


\bigskip 
\noindent \textbf{Overview of Handout Content}:

\begin{enumerate}[noitemsep]
    \item Conceptual (higher-level) notes:
         \begin{enumerate}[noitemsep]
            \item Do the dimensions tested track what you think they're tracking?
                \begin{enumerate}[noitemsep]
                    \item Thoughts
                    \item Methodological responses
                \end{enumerate}
            \item Main vs. alternative hypotheses \\
        \end{enumerate}


    \item Thoughts on other possible experimental paradigms
        \begin{enumerate}[noitemsep]
            \item Evaluating context/task effects
            \item A more naturalistic task?
            \item Switching cost paradigm (cf. Hafri, Trueswell, Strickland 2016)
        \end{enumerate}
\end{enumerate}

\section{Conceptual Considerations}


\subsection{Are the features testing what we think they're testing?}

\subsubsection{Thoughts}
\begin{itemize}
    \item There seems to be a good amount of polysemy in the words chosen
    
    \item In particular, some of the word pairs may not show the desired featural dissociations: 
        \begin{itemize}
             \item 'hug'/'like', are they really different domains? they are conceptually similar---hugging is a consequence of liking
        \end{itemize}

    \item Is Physical vs. Psychological a good (=accessible for naive participants) dimension for testing the ontological category of an event? 
    
    \item Challenge here is to get words with the right features (or show the right dissociation of features) to test what you want.\\
    $\rightarrow$ Do the words in fact have the right features?

    \item We want words that have \textbf{both} features lexically encoded as equally as possible.\\
    $\rightarrow$ Are the features encoded equally in the items?


\end{itemize}


\subsubsection{Methodological Responses}

\begin{itemize}
    \item Norming for core word meaning across several dimensions
    \begin{itemize}
        \item If we quantify these values they can used as control variables in statistical models
        
        \item Set a number of featural norming criterion for evaluating emotionality and extremity of language (Rockledge \& Fazio 2015; Rockledge et al 2018, Evaluative Lexicon)
    \end{itemize}

    \item How consistent are the judgements across speakers? 
    \begin{itemize}
        \item If there is more variability, then the feature is less likely to be a part of the core meaning of the word; likewise, less variability is stronger evidence for the feature being a part of core meaning.
        \item Can be a good way for determining candidate features for core meaning (evidence that denotative aspects of meaning vary less between participants, 
    \end{itemize}
   
    \item Including words that have neutral valence as control items
    
    \item Experiment 1:
    \begin{itemize}
        \item Controlling for log frequency, length, arousal...
        \item Control \& filler items?\\
            Controls as criterion for excluding participants
        \item Counterbalancing key-press between subjects and collecting demo data on handedness
    \end{itemize}
    
    \item Experiment 2:
    \begin{itemize}
        \item Controlling for similarity effects by measuring semantic relatedness
        \item Items matched after more rigorous featural norming
        \item Exp1 results can be used to choose items for Exp2
    \end{itemize}

\end{itemize}



\subsection{Main vs. Alternative Hypotheses}

\noindent Main Hypothesis: \textbf{Affective Primacy Hypothesis}, that the affective component of lexical meaning is prior to the non-affective component of lexical meaning.\\

\begin{itemize}
    \item \textbf{Cognitive Primacy Hypothesis}: it's the non-affective information (ontological category) that's more primary in processing (Nummenmaa et al. 2010)\\
    
    \item \textbf{``Context/Task Primacy Hypothesis''}: which kind of content is primary is determined by context/task demands, i.e., affective contexts favor affective information, while non-affective contexts favor non-affective information (Lai et al. 2012)\\
    
    \item (\textbf{Null Hypothesis}: there are no differences across content types)
    \item (\textbf{Null$^\prime$ Hypothesis}: differences fall on a completely different dimension)
\end{itemize}%

\section{Other potential methodologies}

\subsection{Evaluating context/task as a factor}

\begin{itemize}
    \item Considering alternative hypotheses, I would be interested in looking at the role that context or task demands play in the processing of ontological vs. affective categorization 

    \item Consistent with recent views on probabilistic cognition, essentially context/task demands can be thought of as affecting expectations about upcoming categories

    \item Consistent with views on expectations modulating categorization

    \item Which aspect of meaning is more important to the communicative act at hand?
    
\end{itemize}

\subsection{A more naturalistic task?}

\begin{itemize}
    \item If you're interested in language processing, then using a more naturalistic task
    \item Self-paced readings, where verbs are embedded in sentences
\end{itemize}

\subsection{Switching cost paradigm}

\begin{itemize}
    \item Parallel to event role recognition literature (Hafri, Trueswell, Strickland 2017)
    \item If either valence or ontological category evaluation is automatic process, then there should be a cost when the evaluation category is switched from the one demonstrated
\end{itemize}

            
 





\end{document}
